\documentclass[a4paper]{book}
\usepackage{extarrows}
\usepackage{amsmath,amsthm}
\usepackage{amsmath}
\usepackage{amssymb}
\usepackage{amsfonts}
\usepackage{answers}
\usepackage{eufrak}
\usepackage{eucal}
\usepackage{fancyhdr}
\usepackage{graphicx}
{\setlength\arraycolsep{2pt}
\usepackage{mathrsfs}
\usepackage{graphicx,fancyhdr}
\usepackage{graphicx} 
\usepackage{CJKnumb}
\usepackage{titlesec}
\titleformat{\chapter}{}{}{0em}{\bf\huge}
\usepackage{amsthm}
\usepackage{cite}
\usepackage[utf8]{inputenc}
\usepackage[colorlinks,urlcolor=blue,citecolor=blue,linkcolor=blue]{hyperref}
\usepackage{tikz}
\usepackage{setspace}
\usepackage{xspace}

\newtheorem{theorem}{Theorem}%[section]
\newtheorem{lemma}[theorem]{Lemma}%[section]
\newtheorem{example}{Example}%[section]
\newtheorem*{remark*}{Remark}%[section]
\newtheorem*{note*}{Note}%[section]
\newtheorem{proposition}{Proposition}%[section]
\newtheorem{definition}{Definition}%[section]
\newtheorem{conjecture}{Conjecture}%[section]
\newtheorem{corollary}{Corollary}%[section]
\newtheorem{problem}{Problem}%[section]
\newtheorem{condition}{Condition}%[section]

\renewcommand{\proofname}{\textbf{Proof}}

\newenvironment{ddd}{\begin{rmdef}\rm}{\end{rmdef}}                             
\newenvironment{eee}{\begin{rmexa}\rm}{\end{rmexa}}                             
\newenvironment{rrr}{\begin{rmrem}\rm}{\end{rmrem}}                                   
					                                        
\newenvironment{pf}[1][Proof]{\par\noindent{\em #1}. }{\hfill\framebox(6,6)\par\medskip}

\usepackage{geometry}
 \usepackage[normalem]{ulem}
%\renewcommand{\baselinestretch}{1.5}

\geometry{left=3.5cm,right=3.5cm,top=2.5cm,bottom=2.5cm}

\newcommand{\Poincare}{Poincar\'e\xspace}
\newcommand{\Holder}{Hölder}

\usepackage{graphicx} %picture
\usepackage{float} %picture
\usepackage{subfigure} %picture

\usepackage[nottoc,notlot,notlof]{tocbibind}
%Hölder
%\renewcommand{\baselinestretch}{1.5}
\let\cleardoublepage\clearpage
\begin{document}

\begin{titlepage}\phantom{|}\vspace{0.75in}
\begin{center}
    \underline{THE ISOPERIMETRIC PROBLEM}
\end{center}
\begin{center}
    \underline{}
\end{center}
\vspace{1.5in}%{2.0in}
\begin{center}
    Dissertation submitted at the University of Leicester \\
    in partial fulfilment of the requirements for \\ 
    the degree of Bachelor of Science of Mathematics\\
\end{center}
\vspace{.5in}
\begin{center}
    by
\end{center}
\vspace{.5in}
\begin{center}
    Steven Cheung \\
    Department of Mathematics \\
    University of Leicester \\
\end{center}
\vspace{0.5in}
\begin{center}
    May 2024
\end{center}
\end{titlepage}

\tableofcontents
\thispagestyle{empty}
\chapter*{Declaration}                % The * means no number for this chapter
\pagenumbering{arabic}
\addcontentsline{toc}{chapter}{\hspace{0.2in}Declaration}
All sentences or passages quoted in this project dissertation from other
people's work have been specifically acknowledged by clear cross referencing
to author, work and page(s).  I understand that failure to do this amounts
to plagiarism and will be considered grounds for failure in this module and
the degree examination as a whole.


\bigskip

\noindent
Name: Steven Cheung


\bigskip

\noindent
Signed:


\bigskip

\noindent
Date:


\chapter*{Abstract}
\addcontentsline{toc}{chapter}{\hspace{0.2in}Abstract}
In general, we want the maximum area whose boundary has a specific length. 
\newline
\newline
For the 2-dimensional case.
\newline
\newline
For the 3-dimensional.
\newline
\newline
For the $n$-dimensional.
\newline
\newline
Manifolds?


\chapter*{Introduction}
\addcontentsline{toc}{chapter}{\hspace{0.2in}Introduction}
The isoperimetric  problem, 
\section*{Historical Notes}
\addcontentsline{toc}{section}{\hspace{0.2in}Historical Notes}
Something about historical notes. In the 2 dimensional case, a proof was given by Jakob Sternier, who was Riemann's teacher. 
\section*{Important Preliminaries}
\addcontentsline{toc}{chapter}{\hspace{0.2in}Important Preliminaries}
imp prelims


\chapter{The Isoperimetric Theorems for 2D, 3D and $n$D Cases}
\section{2 Dimensional Case (Plane)}

\section{3 Dimensional Case (Sphere)}

\section{$n$ Dimensional Case ($\mathbb{R}^n$)}

\chapter{Manifolds}
\end{document}